\section{Personalisation} \label{Themes}


\begin{frame}[fragile]{Available Themes}
    Here are the 9 available themes :
    {\footnotesize \begin{itemize}
        \item Black\&White: \testcolor{black}, \testcolor{ur-background-grey}, \testcolor{white}
        \item Red: \testcolor{ur-rot}, \testcolor{urBlock-rot-title}, \testcolor{urBlock-rot-main}
        \item Yellow: \testcolor{ur-yellow}, \testcolor{urBlock-yellow-title}, \testcolor{urBlock-yellow-main}
        \item GreenBrown: \testcolor{ur-greenBrown}, \testcolor{urBlock-greenBrown-title}, \testcolor{urBlock-greenBrown-main}
        \item DarkBlue: \testcolor{ur-darkBlue}, \testcolor{urBlock-darkBlue-title}, \testcolor{urBlock-darkBlue-main}
        \item Orange: \testcolor{ur-orange}, \testcolor{urBlock-orange-title}, \testcolor{urBlock-orange-main}
        \item BlueGreen: \testcolor{ur-blueGreen}, \testcolor{urBlock-blueGreen-title}, \testcolor{urBlock-blueGreen-main}
        \item Green: \testcolor{ur-green}, \testcolor{urBlock-green-title}, \testcolor{urBlock-green-main}
        \item Blue: \testcolor{ur-blue}, \testcolor{urBlock-blue-title}, \testcolor{urBlock-blue-main}
    \end{itemize}}

    change the theme by updating \verb|\themecolor{theme}| in main.tex. \alert{Try to match the theme with your faculty's colour}
    
\end{frame}

%=======================================================================

\begingroup
\themecolor{Black&White}
\begin{frame}[fragile]{Blocks and themes: Black\&White}
    \begin{center}\verb|\themecolor{Black&White}|\end{center}
\begin{columns} % adding [onlytextwidth] the left margins will be set correctly
\begin{column}{0.3\textwidth}
\begin{block}{Standard Blocks}
These have colours coordinated with the general UR theme
\begin{verbatim}
\begin{block}{title}
content...
\end{block}
\end{verbatim}
\end{block}
\end{column}
\begin{column}{0.7\textwidth}
\begin{themedColorBlock}{Colour Blocks}
Block of single colour themed with faculty's colour theme.
\small
\begin{verbatim}
\begin{themedColorBlock}{title}
content...
\end{themedColorBlock}
\end{verbatim}
\end{themedColorBlock}
\begin{themedTitleBlock} {Themed block}
These have colours themed with the titles' colour.
\small
\begin{verbatim}
\begin{themedTitleBlock}{title}
content...
\end{themedblock}
\end{verbatim}
\end{themedTitleBlock}
\end{column}
\end{columns}
\end{frame}
\endgroup

%=======================================================================

\begingroup
\themecolor{Informatik-Psycho}
\begin{frame}[fragile]{Blocks and themes: Informatik-Psycho}
    \begin{center}\verb|\themecolor{Informatik-Psycho}|\end{center}
\begin{columns} % adding [onlytextwidth] the left margins will be set correctly
\begin{column}{0.3\textwidth}
\begin{block}{Standard Blocks}
These have colours coordinated with the general UR theme
\begin{verbatim}
\begin{block}{title}
content...
\end{block}
\end{verbatim}
\end{block}
\end{column}
\begin{column}{0.7\textwidth}
\begin{themedColorBlock}{Colour Blocks}
Block of single colour themed with faculty's colour theme.
\small
\begin{verbatim}
\begin{themedColorBlock}{title}
content...
\end{themedColorBlock}
\end{verbatim}
\end{themedColorBlock}
\begin{themedTitleBlock} {Themed block}
These have colours themed with the titles' colour.
\small
\begin{verbatim}
\begin{themedTitleBlock}{title}
content...
\end{themedblock}
\end{verbatim}
\end{themedTitleBlock}
\end{column}
\end{columns}
\end{frame}
\endgroup

%=======================================================================

\begingroup
\themecolor{Theologie}
\begin{frame}[fragile]{Blocks and themes: Theologie}
\begin{columns} % adding [onlytextwidth] the left margins will be set correctly
\begin{column}{0.3\textwidth}
\begin{block}{Standard Blocks}
These have colours coordinated with the general UR theme
\begin{verbatim}
\begin{block}{title}
content...
\end{block}
\end{verbatim}
\end{block}
\end{column}
\begin{column}{0.7\textwidth}
\begin{themedColorBlock}{Colour Blocks}
Block of single colour themed with faculty's colour theme.
\small
\begin{verbatim}
\begin{themedColorBlock}{title}
content...
\end{themedColorBlock}
\end{verbatim}
\end{themedColorBlock}
\begin{themedTitleBlock} {Themed block}
These have colours themed with the titles' colour.
\small
\begin{verbatim}
\begin{themedTitleBlock}{title}
content...
\end{themedblock}
\end{verbatim}
\end{themedTitleBlock}
\end{column}
\end{columns}
\end{frame}
\endgroup

%=======================================================================

\begingroup
\themecolor{Wirtschafts}
\begin{frame}[fragile]{Blocks and themes: Wirtschafts}
\begin{columns} % adding [onlytextwidth] the left margins will be set correctly
\begin{column}{0.3\textwidth}
\begin{block}{Standard Blocks}
These have colours coordinated with the general UR theme
\begin{verbatim}
\begin{block}{title}
content...
\end{block}
\end{verbatim}
\end{block}
\end{column}
\begin{column}{0.7\textwidth}
\begin{themedColorBlock}{Colour Blocks}
Block of single colour themed with faculty's colour theme.
\small
\begin{verbatim}
\begin{themedColorBlock}{title}
content...
\end{themedColorBlock}
\end{verbatim}
\end{themedColorBlock}
\begin{themedTitleBlock} {Themed block}
These have colours themed with the titles' colour.
\small
\begin{verbatim}
\begin{themedTitleBlock}{title}
content...
\end{themedblock}
\end{verbatim}
\end{themedTitleBlock}
\end{column}
\end{columns}
\end{frame}
\endgroup

%=======================================================================

\begingroup
\themecolor{Medizin}
\begin{frame}[fragile]{Blocks and themes: Medizin}
\begin{columns} % adding [onlytextwidth] the left margins will be set correctly
\begin{column}{0.3\textwidth}
\begin{block}{Standard Blocks}
These have colours coordinated with the general UR theme
\begin{verbatim}
\begin{block}{title}
content...
\end{block}
\end{verbatim}
\end{block}
\end{column}
\begin{column}{0.7\textwidth}
\begin{themedColorBlock}{Colour Blocks}
Block of single colour themed with faculty's colour theme.
\small
\begin{verbatim}
\begin{themedColorBlock}{title}
content...
\end{themedColorBlock}
\end{verbatim}
\end{themedColorBlock}
\begin{themedTitleBlock} {Themed block}
These have colours themed with the titles' colour.
\small
\begin{verbatim}
\begin{themedTitleBlock}{title}
content...
\end{themedblock}
\end{verbatim}
\end{themedTitleBlock}
\end{column}
\end{columns}
\end{frame}
\endgroup


%=======================================================================

\begingroup
\themecolor{Orange}
\begin{frame}[fragile]{Blocks and themes: Orange}
\begin{columns} % adding [onlytextwidth] the left margins will be set correctly
\begin{column}{0.3\textwidth}
\begin{block}{Standard Blocks}
These have colours coordinated with the general UR theme
\begin{verbatim}
\begin{block}{title}
content...
\end{block}
\end{verbatim}
\end{block}
\end{column}
\begin{column}{0.7\textwidth}
\begin{themedColorBlock}{Colour Blocks}
Block of single colour themed with faculty's colour theme.
\small
\begin{verbatim}
\begin{themedColorBlock}{title}
content...
\end{themedColorBlock}
\end{verbatim}
\end{themedColorBlock}
\begin{themedTitleBlock} {Themed block}
These have colours themed with the titles' colour.
\small
\begin{verbatim}
\begin{themedTitleBlock}{title}
content...
\end{themedblock}
\end{verbatim}
\end{themedTitleBlock}
\end{column}
\end{columns}
\end{frame}
\endgroup

%=======================================================================

\begingroup
\themecolor{BlueGreen}
\begin{frame}[fragile]{Blocks and themes: BlueGreen}
\begin{columns} % adding [onlytextwidth] the left margins will be set correctly
\begin{column}{0.3\textwidth}
\begin{block}{Standard Blocks}
These have colours coordinated with the general UR theme
\begin{verbatim}
\begin{block}{title}
content...
\end{block}
\end{verbatim}
\end{block}
\end{column}
\begin{column}{0.7\textwidth}
\begin{themedColorBlock}{Colour Blocks}
Block of single colour themed with faculty's colour theme.
\small
\begin{verbatim}
\begin{themedColorBlock}{title}
content...
\end{themedColorBlock}
\end{verbatim}
\end{themedColorBlock}
\begin{themedTitleBlock} {Themed block}
These have colours themed with the titles' colour.
\small
\begin{verbatim}
\begin{themedTitleBlock}{title}
content...
\end{themedblock}
\end{verbatim}
\end{themedTitleBlock}
\end{column}
\end{columns}
\end{frame}
\endgroup

%=======================================================================

\begingroup
\themecolor{Green}
\begin{frame}[fragile]{Blocks and themes: Green}
\begin{columns} % adding [onlytextwidth] the left margins will be set correctly
\begin{column}{0.3\textwidth}
\begin{block}{Standard Blocks}
These have colours coordinated with the general UR theme
\begin{verbatim}
\begin{block}{title}
content...
\end{block}
\end{verbatim}
\end{block}
\end{column}
\begin{column}{0.7\textwidth}
\begin{themedColorBlock}{Colour Blocks}
Block of single colour themed with faculty's colour theme.
\small
\begin{verbatim}
\begin{themedColorBlock}{title}
content...
\end{themedColorBlock}
\end{verbatim}
\end{themedColorBlock}
\begin{themedTitleBlock} {Themed block}
These have colours themed with the titles' colour.
\small
\begin{verbatim}
\begin{themedTitleBlock}{title}
content...
\end{themedblock}
\end{verbatim}
\end{themedTitleBlock}
\end{column}
\end{columns}
\end{frame}
\endgroup

%=======================================================================

\begingroup
\themecolor{Blue}
\begin{frame}[fragile]{Blocks and themes: Blue}
\begin{columns} % adding [onlytextwidth] the left margins will be set correctly
\begin{column}{0.3\textwidth}
\begin{block}{Standard Blocks}
These have colours coordinated with the general UR theme
\begin{verbatim}
\begin{block}{title}
content...
\end{block}
\end{verbatim}
\end{block}
\end{column}
\begin{column}{0.7\textwidth}
\begin{themedColorBlock}{Colour Blocks}
Block of single colour themed with faculty's colour theme.
\small
\begin{verbatim}
\begin{themedColorBlock}{title}
content...
\end{themedColorBlock}
\end{verbatim}
\end{themedColorBlock}
\begin{themedTitleBlock} {Themed block}
These have colours themed with the titles' colour.
\small
\begin{verbatim}
\begin{themedTitleBlock}{title}
content...
\end{themedblock}
\end{verbatim}
\end{themedTitleBlock}
\end{column}
\end{columns}
\end{frame}
\endgroup

%=======================================================================

\begin{frame}{Adapting the template}
    If you are not familiar with beamer, I strongly advise against modifying the \textbf{Config/style.sty} file.
    
    \vspace{\baselineskip}
    
    If you are not satisfied with any of the available theme, just modify \textbf{Config/colors.sty} as follows:
    \begin{itemize}
        \item Choose one theme (ex: Red)
        \item Modify the colors associated to the theme (ex: \testcolor{\useMainColor}, \testcolor{\useBlockTitleColor} and \testcolor{\useBlockMainColor})
    \end{itemize}
\end{frame}


%=======================================================================

\begin{frame}[fragile]{Using Colours}
\begin{itemize}[<alert@2>]
  \item You can use colours with the
        \verb|\textcolor{<color name>}{text}| command
  \item The colours are defined in the \texttt{Config/colors} package:
  \begin{itemize}
  \item Primary colours: \testcolor{ur-grey} and its sidekick 
  \testcolor{ur-background-grey}
  \item Faculty's colours: see page 10
  \end{itemize}
  \item Do \emph{not} abuse colours: \verb|\emph{}| is usually enough
  \item Use \verb|\alert{}| to bring the \alert<3->{focus} somewhere
  \item <3- | alert@3> If you highlight too much, you don't highlight at all!
\end{itemize}
\end{frame}

%=======================================================================

\begin{frame}[fragile]{Adding images}
\begin{columns}  % adding [onlytextwidth] the left margins will be set correctly
\begin{column}{0.7\textwidth}
Adding images works like in normal \LaTeX:
\begin{block}{Code for Adding Images}
\begin{verbatim}
\usepackage{graphicx}
% ...
\includegraphics[width=\textwidth]
{Config/logo_RGB}
\end{verbatim}
\end{block}
\end{column}
\begin{column}{0.3\textwidth}
\includegraphics[width=\textwidth]{Config/logo_transparent.png}
\end{column}
\end{columns}
\end{frame}

%=======================================================================

\begin{frame}[fragile]{Splitting in Columns}
Splitting the page is easy and common; typically, one side has a picture and the other text:
\begin{columns}  % adding [onlytextwidth] the left margins will be set correctly
\begin{column}{0.6\textwidth}
This is the first column
\end{column}
\begin{column}{0.3\textwidth}
And this the second
\end{column}
\end{columns}
\begin{block}{Column Code}
\begin{verbatim}
\begin{columns} 
    % adding [onlytextwidth] the left margins will be set correctly
    \begin{column}{0.6\textwidth}
        This is the first column
    \end{column}
    \begin{column}{0.3\textwidth}
        And this the second
    \end{column}
    % There could be more!
\end{columns}
\end{verbatim}
\end{block}

\end{frame}
